\documentclass[12pt, a4paper]{article}
\usepackage{mathtools}
\usepackage{blindtext}
\usepackage{graphicx}
\usepackage{xeCJK}
\setCJKmainfont{Noto Serif CJK TC}[
	FontFace={el}{n}{* ExtraLight},
	FontFace={l}{n}{* Light},
	FontFace={m}{n}{* Medium},
	FontFace={sb}{n}{* SemiBold},
	FontFace={b}{n}{* Bold},
	FontFace={eb}{n}{* Black}
]
\setCJKsansfont{Noto Sans CJK TC}[
	FontFace={el}{n}{* Thin},
	FontFace={l}{n}{* Light},
	FontFace={sl}{n}{* DemiLight},
	FontFace={m}{n}{* Medium},
	FontFace={b}{n}{* Bold},
	FontFace={eb}{n}{* Black}
]
\setCJKmonofont{Noto Sans Mono CJK TC}[
	FontFace={b}{n}{* Bold}
]

\title{Hello, \LaTeX\ World}
\author{nevikw39}

\begin{document}
\maketitle
\tableofcontents
\listoftables
\section{hello, world}
This is my first \textsf{\textbf{\TeX}} document.\\
\begin{table}[h!]
    \centering
    \begin{tabular}{l|l}
        ExtraLight & {\fontseries{el}\selectfont 中文} \\ \hline
        Light      & {\fontseries{l}\selectfont 中文}  \\ \hline
        Medium     & {\fontseries{m}\selectfont 中文}  \\ \hline
        Semibold   & {\fontseries{sb}\selectfont 中文} \\ \hline
        Bold       & {\fontseries{b}\selectfont 中文}  \\ \hline
        Black      & {\fontseries{eb}\selectfont 中文} \\
    \end{tabular}
    \caption{CJK Main fonts}
\end{table}
\begin{table}[h!]
    \centering\sffamily
    \begin{tabular}{l|l}
        Thin      & {\fontseries{el}\selectfont 中文} \\ \hline
        Light     & {\fontseries{l}\selectfont 中文}  \\ \hline
        DemiLight & {\fontseries{sl}\selectfont 中文} \\ \hline
        Medium    & {\fontseries{m}\selectfont 中文}  \\ \hline
        Bold      & {\fontseries{b}\selectfont 中文}  \\ \hline
        Black     & {\fontseries{eb}\selectfont 中文} \\
    \end{tabular}
    \caption{CJK Sans fonts}
\end{table}
\begin{table}[h!]
    \centering\ttfamily
    \begin{tabular}{l|l}
        Medium & {\fontseries{m}\selectfont 中文} \\ \hline
        Bold    & {\fontseries{b}\selectfont 中文} \\
    \end{tabular}
    \caption{CJK Mono fonts}
\end{table}
\section{《周易·謙卦》}
\begin{center}
	\includegraphics[width=\textwidth/8]{chiang.png}
\end{center}
亨,君子有終。
\subsubsection{彖傳}
謙,亨,天道下濟而光明,地道卑而上行。天道虧盈而益謙,地道變盈而流謙,鬼神害盈而福謙,人道惡盈而好謙。謙尊而光,卑而不可踰,君子之終也。
\subsubsection{象傳}
地中有山,謙;君子以裒\marginpar{裒,音ㄆㄡˊ,減少}多益寡,稱物平施。
\subsection{初六}
謙謙君子,用涉大川,吉。
\subsubsection{象傳}
謙謙\textsf{君}子,卑以自\textsf{牧}也。\footnote{《諫太宗十思疏》:「念高危,則思謙沖而自牧。」}
\subsection{六二}
鳴謙,貞吉。
\subsubsection{象傳}
鳴謙貞吉,中心得也。
\subsection{九三}
勞謙,君子有終,吉。
\subsubsection{象傳}
勞謙君子,萬民服也。
\subsection{六四}
无不利,撝\marginpar{撝,音ㄏㄨㄟ,謙讓}謙。
\subsubsection{象傳}
无不利,撝謙;不違則也。
\subsection{六五}
不富,以其鄰,利用侵伐,无不利。
\subsubsection{象傳}
利用侵伐,征不服也。
\subsection{上六}
鳴謙,利用行師,征邑國。
\subsubsection{象傳}
鳴謙,志未得也。可用行師,征邑國也。
\pagebreak
\section{Fabonacci Sequence}
\[
	F_0 = 0
\]
\[
	F_1 = 1
\]
\[
	F_n = F_{n-1} + F_{n-2}\ (n \geq 2)
\]
\subsection{Martix}
Let $Fib(n) = \begin{bmatrix} F_n & F_{n-1}\end{bmatrix}$.\\[1cm]
Given $Fib(n) = base \times Fib(n-1)$.\\[1em]
Then $base = \begin{bmatrix} 1 & 1 \\ 1 & 0\end{bmatrix}$.\\
Therefore, $Fib(n) = \begin{bmatrix} 1 & 1\end{bmatrix} \times \begin{bmatrix} 1 & 1 \\ 1 & 0\end{bmatrix}^{n-2}$.\\[10pt]
With \textsl{binary exponentiation}, we can compute \textit{Fabonacci Number} in $\Theta(\log n)$.
\pagebreak
\section{Blind}
\subsection{Text}
\blindtext
\subsection{Itemize}
\blinditemize[5]
\subsection{Enum}
\blindenumerate[5]
\subsection{Description}
\blinddescription[5]
\subsection{Math}
\blindmathpaper
\end{document}
